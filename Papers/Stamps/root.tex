\documentclass[11pt,a4paper]{article}
\usepackage{isabelle,isabellesym}
\usepackage{amssymb}
\usepackage{tikz-cd}

% this should be the last package used
\usepackage{pdfsetup}

% urls in roman style, theory text in math-similar italics
\urlstyle{rm}
\isabellestyle{it}

% for uniform font size
%\renewcommand{\isastyle}{\isastyleminor}

% display anything in a snip without any markup
\newcommand{\Snip}[1]{\isanewline\isanewline}
\newcommand{\EndSnip}{\isanewline\isanewline}
\newcommand{\induct}[1]{}


\begin{document}

\title{GraalVM Stamp Theory}
\maketitle

\begin{abstract}
The GraalVM compiler uses stamps to track type and range
information during program analysis.
Type information is recorded by using distinct subclasses
of the abstract base class \texttt{Stamp},
i.e. \texttt{IntegerStamp} is used to represent an integer type.
Each subclass introduces facilities for tracking range information.
Every subclass of the \texttt{Stamp} class forms a lattice,
together with an arbitrary top and bottom element each sublattice
forms a lattice of all stamps.
This Isabelle/HOL theory models stamps as instantiations of a lattice.
\end{abstract}

\pagebreak

\tableofcontents

\pagebreak

% sane default for proof documents
\parindent 0pt\parskip 0.5ex

% generated text of all theories
\input{Stamp4}

% optional bibliography
%\bibliographystyle{abbrv}
%\bibliography{root}

\end{document}

%%% Local Variables:
%%% mode: latex
%%% TeX-master: t
%%% End:
